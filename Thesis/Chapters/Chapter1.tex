% Chapter 1

\chapter{Title of Chapter 1} % Main chapter title

\label{Chapter1} % For referencing the chapter elsewhere, use \ref{Chapter1} 

%----------------------------------------------------------------------------------------

% Define some commands to keep the formatting separated from the content 
\newcommand{\keyword}[1]{\textbf{#1}}
\newcommand{\tabhead}[1]{\textbf{#1}}
\newcommand{\code}[1]{\texttt{#1}}
\newcommand{\file}[1]{\texttt{\bfseries#1}}
\newcommand{\option}[1]{\texttt{\itshape#1}}

%----------------------------------------------------------------------------------------

\section{Title of Section 1}

Morbi est justo, accumsan nec urna in, pharetra luctus elit. Aenean et orci urna. Praesent et eleifend augue. Integer nec accumsan nisi, vel porttitor odio. Nullam ut magna eu est fermentum accumsan. Aliquam neque lectus, tincidunt quis iaculis ac, luctus eget orci. Ut sit amet tortor lorem.
%----------------------------------------------------------------------------------------


\begin{itemize}
\item \textbf{Item 1)}.\newline
  Contents of Item 1.
\item \textbf{Item 2}.\newline
  Contents of Item 2
\end{itemize}

%----------------------------------------------------------------------------------------


\section{\nohyphens{Title of Section 2}}
%nohyphens prevents line breaking on title (if needed)
Integer sodales ultricies odio et feugiat. Nunc consequat dui vel justo ultrices tempus. Morbi fringilla urna sed tempus condimentum. Nullam congue arcu sollicitudin elementum dictum. Sed elementum nunc a pretium rhoncus. Mauris pretium condimentum augue, vitae ultrices nulla volutpat id. Integer consequat risus vulputate metus viverra semper. Donec euismod imperdiet urna, non lacinia dolor. Donec dapibus cursus ex eget consectetur. Aliquam ac magna auctor, ullamcorper nibh ut, sodales leo. Proin metus mauris, venenatis sit amet imperdiet ac, lobortis in neque.
\newline

\vspace{0.4cm}
\begin{figure}[h]
  \centering
  \includegraphics[scale=0.3,angle=0]{Figures/Electron.pdf}
  \caption{Figure description}
  \label{electron}
\end{figure}


As seen on figure \ref{electron}


\section{Title of Section 3}
Nulla ultricies ante vehicula, porta leo sed, lobortis purus. Maecenas et tellus massa. Duis cursus, elit a iaculis faucibus, eros libero accumsan diam, quis cursus massa est eu metus. Pellentesque habitant morbi tristique senectus et netus et malesuada fames ac turpis egestas. In hac habitasse platea dictumst. Vestibulum ante ipsum primis in faucibus orci luctus et ultrices posuere cubilia Curae; Etiam luctus dapibus diam sit amet elementum. Phasellus eget velit in eros convallis sagittis. Morbi tempus, eros ut rutrum efficitur, sem augue facilisis augue, ullamcorper egestas est erat sed ligula. Mauris nibh purus, porttitor eu semper non, lacinia sit amet elit. Vestibulum nec erat non erat ornare placerat non nec eros. Sed molestie ultrices mauris quis sagittis.
\cite{Kogelnik, saleh2007fundamentals, Silfvast1996b, siegman1986lasers, amnon1989quantum}.
